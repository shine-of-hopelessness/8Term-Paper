OpenCL  — фреймворк для 
написания параллельных программ для различных платформ, включающих в
себя центральные процессоры и графические ускорители . В фреймворк OpenCL входят
язык программирования для акселератора, который базируется на стандарте C99, и
интерфейс для программы выполняющихся на центральном процессоре. OpenCL является
полностью открытым стандартом, разрабатывается и поддерживается некоммерческим консорциумом Khronos Group, 
в который входят много крупных компаний, включая Apple, AMD, Intel, nVidia, Sun Microsystems, Sony 
Computer Entertainment и другие.

Программная модель OpenCL позволяет программисту описывать функции, которые будут
параллельно выполнены на некотором акселераторе или наборе акселераторов, доступных на
данной машине.
В основе программной модели OpenCL лежит понятия ядра (kernel). Ядро — это функция,
которая будет выполнена параллельно на акселераторе определенным количеством
потоков. Ядра создаются, компилируются и запускаются программой на хосте, с помощью
вызовов функций библиотеки OpenCL. Запуск нужного ядра с указанием требуемого количества потоков
осуществляется с помощью вызова функции библиотеки OpenCL на центральном
процессоре. Таким образом, программа состоит из последовательного кода, который выполняется
на центральном процессоре и параллельного кода, которые выполняется на
акселераторе.
