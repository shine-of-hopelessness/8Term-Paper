OpenCL  — фреймворк для 
написания компьютерных программ, связанных с параллельными вычислениями на различных графических 
 и центральных процессорах . В фреймворк OpenCL входят язык программирования, 
который базируется на стандарте C99, и интерфейс программирования приложений. OpenCL 
обеспечивает параллелизм на уровне инструкций и на уровне данных и является реализацией техники GPGPU. 
OpenCL является полностью открытым стандартом, его использование не облагается лицензионными отчислениями.

Цель OpenCL состоит в том, чтобы дополнить OpenGL и OpenAL, которые являются открытыми отраслевыми 
стандартами для трёхмерной компьютерной графики и звука, пользуясь возможностями GPU. OpenCL разрабатывается 
и поддерживается некоммерческим консорциумом Khronos Group, в который входят много крупных компаний, 
включая Apple, AMD, Intel, nVidia, Sun Microsystems, Sony Computer Entertainment и другие.

Программная модель OpenCL
\cite{OpenCL:Specification, OpenCL:programming-guide, OpenCL:jump-start}
позволяет программисту определять функции, называемые
ядрами (kernels), которые будут выполнены параллельно на устройстве,
поддерживающем данную программную модель.

Ядра создаются, компилируются и запускаются программой на хосте, с помощью
вызовов функций библиотеки OpenCL.

Таким образом, программа состоит из последовательного кода, который выполняется
на хосте и параллельного кода, которые выполняется на устройстве.