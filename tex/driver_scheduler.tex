Планировщик расположен в драйвере для
устройства и работает соответственно на центральном прцоессоре.
Непосредственное общение с устройством сократилось до запуска задач, с помощью
записи некоторых значений по определённым адресам. В остальном, драйвер по
необходимости вызывает функции планировщика и скрывает от него общение с
устройством. В обязанности планировщика входят следующие вещи:
\begin{enumerate}
  \item Приём новых вычислительных задач для выполнения,их сериализация и
  запуск.
  \item Приём новых задач DMA-транзакций, их сериализация и запуск
  \item Информирование драйвера о проблемах при запуске задачи.
  \item Обеспечение корректности выдачи уникальных идентификаторов ядрам.
\end{enumerate}

Экземпляр планировщика создаётся при загрузке модуля драйвера. При добавлении в
систему устройства, для него создаётся соответствующая структура. Далее драйвер
должен задать планировщику сколько и какие ядра будут на устройстве, планировщик
сам назначит им набор уникальных, подряд идущих номеров и вернёт первое значение
драйверу. Теперь возможны запуски задач. У каждого ядра есть своя очередь задач,
на выполнение. Кроме того есть отдельная очередь под
DMA-транзакции.  Планировщик только выполняет запуск задач. Драйвер сам
отслеживает окончание работы и сообщает об этом планировщику, чтобы тот запустил
следующую задачу(если таковая имеется). В случае возникновения ошибки при
запуске, планировщик сообщает об этом драйверу и пытается запустить следующую
задачу. При удалении устройства, удаляется соответствующая структура, все ядра и
очереди задач на них. В процессе работы возможно переустановление ядер для
устройства, в этом случае старые ядра и очереди задач на них удаляются. 
