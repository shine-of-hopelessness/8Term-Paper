В рамках данной работы была проведена разработка планировщика заданий системы
автоматизации программирования для FPGA. Были предложены две реализации, одна
размещала планировщик на софтверном процессоре устройства, другая располагала
его в драйвере устройства. Текущая версия поддерживает несколько устройств с
уникальным набором ядер, очередей задач для них и DMA-очередей. 

В рамках
планировщика в будущем планируется использовать механизм частичной реконфигурации ППВМ для
замены ядер, работающих на FPGA без полного перепрограммирования FPGA
\cite{partial-reconfiguration-report}. Кроме того, планируется реализации
конвееризации для имеющихся ядер с перенаправлением вывода одного ядра на вход
другого, с целью извлечения дополнительного параллелизма.

