Реализованная в рамках данной работы система удовлетворяет исходной постановке
задачи, предоставляя программисту стандартный набор операций для работы с ППВМ
как OpenCL акселераторами. Полученная система тестировалась на ППВМ Xilinx
Virtex-6 (XC6VLX240T)\cite{virtex-6-overview}.

Числовые характеристики работы системы приведены в таблице \ref{results-table}.
\begin{table}[h!]
\caption{Характеристики работы системы}
\begin{center}
\begin{tabular}{|c|c|}
\hline
Скорость записи в память устройства & $986.2$ Мб/сек \\ \hline
Скорость чтения из памяти устройства & $1246.9$ Мб/сек \\ \hline
Задержка запуска ядра & $\sim 7-8$ мкс \\ \hline
Задержка ожидания некоторого события & $\sim 8-10$ мкс \\ \hline
Время запуска ядра & $\sim 100-150$ мкс \\ \hline
Задержка запуска обмена данными & $\sim 8-10$ мкс. \\ \hline
Время перепрограммирования устройства & $\sim 25$ сек \\\hline
\end{tabular}
\end{center}
\label{results-table}
\end{table}

При этом, необходимо отличать задержку запуская ядра (время, прошедшее между
вызовом \texttt{clEnqueueNDRangeKernel} и возвратом из нее) и время запуская
ядра, включающее в себя время выполнения всех необходимых операций в
параллельном потоке очереди команд.