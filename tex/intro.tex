Основным вычислительным устройством в большинстве современных компьютеров
является процессор общего назначения. Он обладает некоторым
заранее определённым набором команд, на которые и разбивается любая задача.
Основным способом увеличения производительности подобных устройств является
повышение количества выполнений данных операций в единицу времени. Для
вычислительно ёмких задач используются системы из таких процессоров и коммуникации между ними. 
Другим способом ускорения вычислений является создание специализированных
процессоров для конкретной задачи. Аппаратная реализация позволяет получить существенный выигрыш
производительности, однако практическая стоимость подобных устройств очень
велика и они ограничены решением только одной задачи. В связи с этим сначала
возникла, а потом реализовалась идея программируемых логических устройств,
которые позволяют пользователю с помощью некоторого языка описания настроить
устройство для решения конкретной задачи. Хотя вычислительная мощность подобных плат и уступает
чисто аппаратной реализации, возможность
перепрограммирования, скорость разработки и ввода в производство, а также
стоимость делают их очень привлекательными. Решение какой - то конкретной
проблемы может как целиком быть перенесено на такое устройство(или систему из
них), так и выполняться на обычном процессоре, с использованием устройства в
качестве ускорителя для некоторых частей программы. На сегодняшний день наиболее
мощными представителями данного семейства устройств являются FPGA. К сожалению,
программирование подобных устройств является трудным для неспециалистов. В связи
с этим предложена концепция автоматизации вычислений на FPGA, основываясь на
стандарте OpenCL. В рамках данной задачи необходимо реализовать некий диспетчер
задач для взаимодействия между FPGA и центральным процессором.
